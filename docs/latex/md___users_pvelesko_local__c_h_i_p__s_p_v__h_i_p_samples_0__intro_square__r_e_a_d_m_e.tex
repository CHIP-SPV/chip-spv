Simple test below is an example, shows how to use hipify-\/perl to port CUDA code to HIP\+:


\begin{DoxyItemize}
\item Add hip/bin path to the PATH
\end{DoxyItemize}


\begin{DoxyCode}{0}
\DoxyCodeLine{\$ export PATH=\$PATH:[MYHIP]/bin}

\end{DoxyCode}



\begin{DoxyItemize}
\item Define environment variable
\end{DoxyItemize}


\begin{DoxyCode}{0}
\DoxyCodeLine{\$ export HIP\_PATH=[MYHIP]}

\end{DoxyCode}



\begin{DoxyItemize}
\item Build executible file
\end{DoxyItemize}


\begin{DoxyCode}{0}
\DoxyCodeLine{\$ cd \string~/hip/samples/0\_Intro/square}
\DoxyCodeLine{\$ make}
\DoxyCodeLine{/home/user/hip/bin/hipify-\/perl square.cu > square.cpp}
\DoxyCodeLine{/home/user/hip/bin/hipcc  square.cpp -\/o square.out}
\DoxyCodeLine{/home/user/hip/bin/hipcc -\/use-\/staticlib  square.cpp -\/o square.out.static}

\end{DoxyCode}

\begin{DoxyItemize}
\item Execute file 
\begin{DoxyCode}{0}
\DoxyCodeLine{\$ ./square.out}
\DoxyCodeLine{info: running on device Navi 14 [Radeon Pro W5500]}
\DoxyCodeLine{info: allocate host mem (  7.63 MB)}
\DoxyCodeLine{info: allocate device mem (  7.63 MB)}
\DoxyCodeLine{info: copy Host2Device}
\DoxyCodeLine{info: launch 'vector\_square' kernel}
\DoxyCodeLine{info: copy Device2Host}
\DoxyCodeLine{info: check result}
\DoxyCodeLine{PASSED!}

\end{DoxyCode}
 
\end{DoxyItemize}